%\documentclass[linenumbers, RNAAS, trackchanges]{aastex631}
\documentclass{aastex631}
%\documentclass[manuscript]{aastex631}
%\documentclass[a4paper, 10pt]{article}

\usepackage[utf8]{inputenc}
\usepackage{hyperref}           % hrefs
\usepackage{natbib}             % for bibliography
\usepackage{float}              % figure positioning
\usepackage{svg}                % used for SVG images
\usepackage{graphicx}           % used for non-SVG images
\usepackage{amsmath}

% Search Query Metadata
\shorttitle{Logistic Map}
\shortauthors{Corresponding Author's Last Name, et al.}

% Hyperlink setup
\hypersetup{
colorlinks=true,
linkcolor=blue,
urlcolor=blue
}

\begin{document}
\title{Bifurcation and Chaos in the Logistic Map}
% [] is for ORCiD
%\correspondingauthor{Main Author}
%\email{author1@email.com, author2@email.com, author3@email.com}
\author{Lily Nguyen}
\affiliation{Department of Physics, The University of Texas at Austin\\
Austin, TX 78712, USA}

\author{Raymond Vu}
\affiliation{Department of ---, The University of Texas at Austin\\
Austin, TX 78712, USA}

%\thanks{These authors contributed equally to this work.}

%\collaboration{2}{CS 323E}
% or 
%\nocollaboration{0}

% 250 word limit for abstract
\begin{abstract}
\noindent The logistic map is a simple nonlinear difference equation that displays
various behaviors ranging from stable fixed points to periodic oscillations
and chaotic dynamics. For this project, we generated generated the bifurcation
diagram of the logistic map over the parameter range $2.4< \mu < 4.0$. We
followed the standard approach of discarding transient iterations to remove
dependence on initial conditions and recording the long-term dynamics of $x$
for each $\mu$. The resulting bifurcation diagram illustrates the progression from
stability to successive period-doubling behavior and the eventual onset of
chaos. The transition occurs near $\mu\approx3.57$, which is consistent with
Feigenbaum's universal period-doubling scenario \cite{feigenbaum}. Overall, 
our results agree with the mathematical analysis of the logistic map 
\cite{bubolo} and highlight the importance of visualization for understanding
nonlinear systems \cite{boeing}.



\end{abstract}

% Use astrothesaurus numbers in place of num
\keywords{logistic map --- bifurcation diagram --- chaos --- nonlinear dynamics}
\section{Introduction} \label{sec:intro}
This is the intro section.

\section{Data} \label{sec:data}
This is the data section.

Maybe add some lists/tables/whatever is needed.



\section{Information about Observations}
Write info about observations here.

\subsection{Specific Topic Pt 1} \label{sec:subtopic1}
topic 1

\subsection{Specific Topic Pt 2} \label{sec:subtopic2}
topic 2

\begin{figure}[H]
    \centering
    \includegraphics[width=0.9\linewidth]{bifurcation.png}
    \caption{The logistic map bifurcation diagram for $2.4\leq \mu \leq 4.0$. Stable fixed points exist for low values of $\mu$, followed by successive period-doubling bifurcations that lead to chaotic behavior near $\mu\approx 3.57$.}
    \label{fig:bifurcation}
\end{figure}


\subsection{Specific Topic Pt 3} \label{sec:subtopic3}
topic 3

\section{Results} \label{sec:results}
write about the results here.


\section{Summary and Conclusion} \label{sec:summary}
Put the summary and conclusions here.


\subsubsection{References}
This is me citing \citet{boeing} in-text. This is me citing Bubolo at the end of this sentence \citep{bubolo}.

\newpage
\bibliographystyle{aasjournal}
%\bibliographystyle{plainnat}
\bibliography{refs}

\end{document}
